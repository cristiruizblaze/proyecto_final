\documentclass{article}
\usepackage[english]{babel}
\usepackage[utf8]{inputenc}
\usepackage{anysize}

\begin{document}
\marginsize{2cm}{2cm}{2cm}{2cm}

%%%%%%%%%%%%%%%%%%%%%%%%%%%%%%%%%%%%%%%%%%%%%%%%%%%%%%%%%%%%%%%%%%%%%
%% Authors
%%%%%%%%%%%%%%%%%%%%%%%%%%%%%%%%%%%%%%%%%%%%%%%%%%%%%%%%%%%%%%%%%%%%%
\author{Cristina Ruiz}
%Department of Chemistry and Physics, Research centre CIAIMBITAL, Ctra. Sacramento, s/n, 04120 Almería, Spain.

\author{Álvaro Raya-Barón}
%Department of Chemistry and Physics, Research centre CIAIMBITAL, Ctra. Sacramento, s/n, 04120 Almería, Spain.
%Both authors have equally contributed

\author{Manuel A. Ortuño}
%Institute of Chemical Research of Catalonia (ICIQ), The Barcelona Institute of Science and Technology (BIST), Av. Països Catalans 16, 43007 Tarragona, Spain.
%mortuno@iciq.es

\author{Ignacio Fernández}
%ifernan@ual.es
%Department of Chemistry and Physics, Research centre CIAIMBITAL, Ctra. Sacramento, s/n, 04120 Almería, Spain.

%%%%%%%%%%%%%%%%%%%%%%%%%%%%%%%%%%%%%%%%%%%%%%%%%%%%%%%%%%%%%%%%%%%%%
%% Title
%%%%%%%%%%%%%%%%%%%%%%%%%%%%%%%%%%%%%%%%%%%%%%%%%%%%%%%%%%%%%%%%%%%%%
\title{Accelerating role of deaggregation agents in lithium-catalysed hydrosilylation of carbonyl compounds} %Electronic supplementary information (ESI) available: Kinetic plots, NMR spectra, computational details, intermediate and transition state structures. See DOI: 10.1039/d0dt01540g
\maketitle

%%%%%%%%%%%%%%%%%%%%%%%%%%%%%%%%%%%%%%%%%%%%%%%%%%%%%%%%%%%%%%%%%%%%%
%% Manuscript
%%%%%%%%%%%%%%%%%%%%%%%%%%%%%%%%%%%%%%%%%%%%%%%%%%%%%%%%%%%%%%%%%%%%%


	%%%%%%%%%%%%%%%%%%%%%%%%%%%%%%%%%%%%%%%%%%%%%%%%%%%%%%%%%%%%%%%%%%%%%
	%% Abstract
	%%%%%%%%%%%%%%%%%%%%%%%%%%%%%%%%%%%%%%%%%%%%%%%%%%%%%%%%%%%%%%%%%%%%%
	
	\begin{abstract}
		A combined computational and experimental approach demonstrates the accelerating role of deaggregation agents, especially HMPA, in the Li-catalysed hydrosilylation of acetophenone in THF solution under very mild conditions.
	\end{abstract}
	
	%%%%%%%%%%%%%%%%%%%%%%%%%%%%%%%%%%%%%%%%%%%%%%%%%%%%%%%%%%%%%%%%%%%%%
	%% Introduction
	%%%%%%%%%%%%%%%%%%%%%%%%%%%%%%%%%%%%%%%%%%%%%%%%%%%%%%%%%%%%%%%%%%%%%
	
	\section{Introduction}
	The reduction of carbonyl groups into alcohols is of wide interest in synthetic chemistry and therefore, constant efforts are	put into developing new efficient methodologies and perfecting
	the existing ones. Catalytic hydrosilylation1 has emerged as a convenient method as it operates under mild conditions and	combines an exceptional reducing capability with a high	selectivity that can be finely tuned via catalyst design. Many Earth-abundant first-row transition metals, specially iron,2,3
	have been tested in catalytic hydrosilylation of carbonyl compounds, as they are usually more environmentally-friendly and less toxic than their second- and third-row counterparts. Alkali
	metal salts have also been explored as alternative catalysts, initially by the groups of Corriu4 and Hosomi,5 and later by Beller6 and Nikonov,7 among others. Such compounds have been employed to promote hydrosilylation due to their basic character via formation of a pentacoordinated hydridosilicate,6–8 usually neglecting any relevant role of the alkali cation in the reaction mechanism. We recently reported	the hydrosilylation of carbonyl compounds catalysed by
	lithiated hydrazones.9 However, full understanding at atomic level of detail is still needed for the rational design of catalysts and reaction conditions.
	Herein we join computational and experimental efforts to understand and optimise processes catalysed by alkali–metal	amides.10 Following theoretical guidance, we demonstrate how deaggregation agents (DAs) can efficiently accelerate hydrosilylation of carbonyl compounds in the presence of readily available lithium amides (Scheme 1) under very mild conditions such as room temperature and very low catalyst loading.
		

	
\end{document}